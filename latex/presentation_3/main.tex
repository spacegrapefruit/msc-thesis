\documentclass[svgnames, 12pt]{beamer}

\usepackage[utf8]{inputenc}
\usepackage[lithuanian,english]{babel}
\usepackage[T1]{fontenc}
\usepackage{lmodern}
\usepackage{amsmath}
\usepackage{amssymb}
\usepackage{booktabs}
\usepackage{xcolor}
\usepackage{subfig}
\usepackage{graphicx}
\usepackage{hyperref}
\usepackage{multirow}

% --- THEME & COLORS ---
\definecolor{mifcolor}{RGB}{0, 71, 127}
\definecolor{dimgr}{RGB}{105, 105, 105}
\definecolor{sky}{RGB}{0, 191, 255}
\setbeamercolor{alerted text}{fg=red,bg=sky}
\newcommand{\boxalert}[1]{{%
      \usebeamercolor{alerted
        text}\colorbox{bg}{\alert{#1}}%
    }}

\mode<presentation>{
  \usetheme{Madrid}
  \usecolortheme[named=mifcolor]{structure}
  \setbeamertemplate{footline}
  {%
    \leavevmode%
    \hbox{
      \begin{beamercolorbox}[wd=.3\paperwidth,ht=2.5ex,dp=1.125ex,leftskip=.3cm
          plus1fill,rightskip=.3cm]{author in
          head/foot}%
        \usebeamerfont{author in head/foot}\insertshortauthor
        \hfill
      \end{beamercolorbox}%

      \begin{beamercolorbox}[wd=.2\paperwidth,ht=2.5ex,dp=1.125ex,leftskip=.3cm,
          rightskip=.3cm plus1fil]{institute in
          head/foot}%
        \usebeamerfont{institute in
          head/foot}\insertshortinstitute
      \end{beamercolorbox}%

      \begin{beamercolorbox}[wd=.2\paperwidth,ht=2.5ex,dp=1.125ex,leftskip=.3cm,
          rightskip=.3cm plus1fil]{date in
          head/foot}%
        \usebeamerfont{date in head/foot}\insertshortdate
      \end{beamercolorbox}%

      \begin{beamercolorbox}[wd=.3\paperwidth,ht=2.5ex,dp=1.125ex,leftskip=.3cm,
          rightskip=.3cm plus1fil]{title in
          head/foot}%
        \usebeamerfont{title in
          head/foot}\insertshorttitle\hfill p.
        \insertframenumber\enspace of
        \inserttotalframenumber\enspace
      \end{beamercolorbox} }%
    \vskip0pt%
  }
}

\title[Multi-Speaker TTS Strategies]{Data Selection Strategies for Multi-Speaker TTS in Lithuanian}
\subtitle{Progress Report 3}
\author[A. J. Smoliakov]{Aleksandr Jan Smoliakov\inst{1}\\
  \vspace{0.5em}
  \small{Supervisor: Dr.~Gerda Ana Melnik-Leroy}\\
  \small{Scientific Advisor: Dr.~Gražina Korvel}}
\institute[VU MIF]{\inst{1} Vilnius University, Faculty of Mathematics and
  Informatics}
\date{2025--12--16}

\begin{document}

\begin{frame}
  \includegraphics[scale=0.15]{MIF Garamond-logo.png}
  \hfill
  \includegraphics[scale=0.15]{Logo_spalvotas.eps}

  \titlepage
\end{frame}

% \begin{frame}{Table of Contents}
%   \tableofcontents
% \end{frame}

\section{Introduction}

\begin{frame}{Recap \& Experimental design}
  \textbf{Liepa 2 challenge:}
  \begin{itemize}
    \item 1000 hours of audio, distributed across 2,621 speakers.
    \item Most speakers have < 30 minutes of data.
    \item Standard TTS requires 10--20 hours single-speaker data.
    \item \textbf{Core RQ}: Under a fixed data budget, what is the optimal data selection strategy for Lithuanian multi-speaker TTS\@?
  \end{itemize}

  \vspace{1em}

  \begin{block}{Three datasets, fixed trainset budget (22.5~h audio)}
    \begin{itemize}
      \item \textbf{Depth}: 30 speakers, 45 min/speaker
      \item \textbf{Balance}: 60 speakers*, 22.5 min/speaker
      \item \textbf{Breadth}: 180 speakers*, 7.5 min/speaker
    \end{itemize}
  \end{block}

  \vspace{0.5em}
  \footnotesize{*Speakers are nested (30 $\subset$ 60 $\subset$ 180) and gender-balanced (50/50).}
\end{frame}

\begin{frame}{Model architectures}
  Two distinct acoustic model architectures were trained on all three subsets (6 experiments total).

  \begin{columns}[T]
    \begin{column}{0.48\textwidth}
      \textbf{Tacotron~2}
      \begin{itemize}
        \item Autoregressive sequence-to-sequence model.
        \item Trained for 200 epochs (on each subset) --- until convergence.
      \end{itemize}

    \end{column}
    \begin{column}{0.48\textwidth}
      \textbf{Glow-TTS}
      \begin{itemize}
        \item Flow-based generative model.
        \item Trained for 400 epochs (on each subset) --- until convergence.
      \end{itemize}

    \end{column}
  \end{columns}

  \vspace{1em}
  \textbf{Vocoder:} Pre-trained \textbf{HiFi-GAN~v2} (frozen) used for all models.
\end{frame}

\section{Results}

\begin{frame}{Objective results}
  Metrics evaluated on held-out test set (60 sentences).

  \begin{table}[]
    \centering
    \small
    \begin{tabular}{lccc}
      \toprule
      \textbf{Model}              & \textbf{Speakers (N)} & \textbf{MCD (dB)} & \textbf{F0 RMSE (Hz)} \\
      \midrule
      \multirow{3}{*}{Tacotron 2} & 30                    & 9.58              & 31.28                 \\
                                  & \textbf{60}           & \textbf{9.55}     & \textbf{30.49}        \\
                                  & 180                   & 9.63              & 31.06                 \\
      \midrule
      \multirow{3}{*}{Glow-TTS}   & \textbf{30}           & \textbf{9.90}     & 37.86                 \\
                                  & 60                    & 10.00             & 36.18                 \\
                                  & 180                   & 9.98              & \textbf{35.69}        \\
      \bottomrule
    \end{tabular}
    \caption{Tacotron~2 moderately, but consistently outperforms Glow-TTS in spectral and pitch accuracy (lower is better).}
  \end{table}

  \textbf{Observation:} Both models were surprisingly insensitive to data composition in terms of objective metrics.
\end{frame}

\begin{frame}{Subjective evaluation (MOS)}
  \textbf{Methodology:}
  \begin{itemize}
    \item 60 sentences (being) rated by 21 Native Lithuanian speakers.
    \item Latin Square Design to mitigate bias.
    \item Scale: 1 (Bad) to 5 (Excellent).
    \item Tacotron~2 significantly outperforms Glow-TTS in naturalness.
  \end{itemize}

  \begin{table}[]
    \centering
    \begin{tabular}{lcc}
      \toprule
      \textbf{Configuration}         & \textbf{Tacotron 2 MOS} & \textbf{Glow-TTS MOS} \\
      \midrule
      30 Speakers (Depth)            & 3.39                    & 2.40                  \\
      \textbf{60 Speakers (Balance)} & \textbf{3.48}           & 2.39                  \\
      180 Speakers (Breadth)         & 3.27                    & 2.16                  \\
      \bottomrule
    \end{tabular}
  \end{table}
\end{frame}

\begin{frame}{MOS by speaker}
  \begin{table}[]
    \centering
    \caption{Average MOS per speaker across all models.}\label{tab:speaker_mos}
    \begin{tabular}{lcc}
      \toprule
      \textbf{Speaker ID} & \textbf{Tacotron~2 MOS}   & \textbf{Glow-TTS MOS} \\
      \midrule
      AS009               & \underline{\textbf{4.22}} & \textbf{2.68}         \\
      IS031               & 3.28                      & 2.28                  \\
      IS038               & 3.60                      & 2.58                  \\
      MS052               & 2.19                      & 1.94                  \\
      VP131               & 2.54                      & 2.00                  \\
      VP427               & 3.18                      & 1.64                  \\
      \bottomrule
    \end{tabular}
  \end{table}
\end{frame}

\begin{frame}{Visual analysis}
  \begin{figure}[ht]
    \centering
    \subfloat[Tacotron 2 (60 speakers)]{
      \includegraphics[width=0.45\textwidth]{mel_sp_taco2.png}
    }
    \hfill
    \subfloat[Glow-TTS (60 speakers)]{
      \includegraphics[width=0.45\textwidth]{mel_sp_glow.png}
    }
    \caption{Mel-spectrograms generated by Tacotron~2 and Glow-TTS for the same input text.}\label{fig:spectrogram_comparison}
  \end{figure}

  \textbf{Spectrogram comparison:}
  Tacotron 2 generates finer spectral details and more dynamic pitch contours compared to the ``flatter'' output of Glow-TTS\@.
\end{frame}

\section{Discussion \& Conclusion}

\begin{frame}{Discussion \& Conclusions}
  \begin{itemize}
    \item Tacotron 2 consistently produced more natural speech than Glow-TTS across all
          data strategies.
    \item Overall, data composition had a limited effect on objective metrics for both
          models, and only a moderate effect on subjective naturalness.
    \item All models' synthesis quality strongly depended on individual speaker
          characteristics.
    \item Using 7.5 minutes per speaker (180 speakers) is viable for intelligible
          multi-speaker TTS in low-resource settings.
  \end{itemize}
\end{frame}

\begin{frame}{Progress summary}
  \begin{columns}[T]
    \begin{column}{0.5\textwidth}
      \textbf{Completed:}
      \begin{itemize}
        \item[\textcolor{green}{\checkmark}] Trained 6 multi-speaker TTS models
        \item[\textcolor{green}{\checkmark}] Conducted objective evaluations
        \item[\textcolor{green}{\checkmark}] Preliminary subjective results and drafted findings
      \end{itemize}
    \end{column}
    \begin{column}{0.5\textwidth}
      \textbf{Ongoing:}
      \begin{itemize}
        \item[\textcolor{orange}{\bullet}] Subjective evaluation (MOS study)
        \item[\textcolor{orange}{\bullet}] Analysis of subjective results
        \item[\textcolor{orange}{\bullet}] Writing thesis draft
      \end{itemize}
    \end{column}
  \end{columns}
\end{frame}

\begin{frame}{Thank You!}
  \begin{center}
    \vspace{2em}

    \Large Thank you for your attention!
  \end{center}
\end{frame}

\end{document}
