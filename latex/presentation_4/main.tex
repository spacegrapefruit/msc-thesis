\documentclass[svgnames, 12pt]{beamer}

\usepackage[utf8]{inputenc}
\usepackage[lithuanian,english]{babel}
\usepackage[T1]{fontenc}
\usepackage{lmodern}
\usepackage{amsmath}
\usepackage{amssymb}
\usepackage{booktabs}
\usepackage{xcolor}
\usepackage{subfig}
\usepackage{graphicx}
\usepackage{hyperref}
\usepackage{multirow}

% --- THEME & COLORS ---
\definecolor{mifcolor}{RGB}{0, 71, 127}
\definecolor{dimgr}{RGB}{105, 105, 105}
\definecolor{sky}{RGB}{0, 191, 255}
\setbeamercolor{alerted text}{fg=red,bg=sky}
\newcommand{\boxalert}[1]{{%
      \usebeamercolor{alerted
        text}\colorbox{bg}{\alert{#1}}%
    }}

\mode<presentation>{
  \usetheme{Madrid}
  \usecolortheme[named=mifcolor]{structure}
  \setbeamertemplate{footline}
  {%
    \leavevmode%
    \hbox{
      \begin{beamercolorbox}[wd=.3\paperwidth,ht=2.5ex,dp=1.125ex,leftskip=.3cm
          plus1fill,rightskip=.3cm]{author in
          head/foot}%
        \usebeamerfont{author in head/foot}\insertshortauthor
        \hfill
      \end{beamercolorbox}%

      \begin{beamercolorbox}[wd=.2\paperwidth,ht=2.5ex,dp=1.125ex,leftskip=.3cm,
          rightskip=.3cm plus1fil]{institute in
          head/foot}%
        \usebeamerfont{institute in
          head/foot}\insertshortinstitute
      \end{beamercolorbox}%

      \begin{beamercolorbox}[wd=.2\paperwidth,ht=2.5ex,dp=1.125ex,leftskip=.3cm,
          rightskip=.3cm plus1fil]{date in
          head/foot}%
        \usebeamerfont{date in head/foot}\insertshortdate
      \end{beamercolorbox}%

      \begin{beamercolorbox}[wd=.3\paperwidth,ht=2.5ex,dp=1.125ex,leftskip=.3cm,
          rightskip=.3cm plus1fil]{title in
          head/foot}%
        \usebeamerfont{title in
          head/foot}\insertshorttitle\hfill p.
        \insertframenumber\enspace of
        \inserttotalframenumber\enspace
      \end{beamercolorbox} }%
    \vskip0pt%
  }
}

\title[Multi-Speaker TTS Strategies]{Training Data Selection Strategies for Multi-Speaker TTS in Lithuanian}
\subtitle{Pre-defense Presentation}
\author[A. J. Smoliakov]{Aleksandr Jan Smoliakov\inst{1}\\
  \vspace{0.5em}
  \small{Supervisor: Dr.~Gerda Ana Melnik-Leroy}\\
  \small{Scientific Advisor: Dr.~Gražina Korvel}}
\institute[VU MIF]{\inst{1} Vilnius University, Faculty of Mathematics and
  Informatics}
\date{2025--12--22}

\begin{document}

\begin{frame}
  \includegraphics[scale=0.15]{MIF Garamond-logo.png}
  \hfill
  \includegraphics[scale=0.15]{Logo_spalvotas.eps}

  \titlepage
\end{frame}

% Goal & Novelty section
\section{Goal \& novelty}

\begin{frame}{Research aim \& novelty}
  \textbf{Problem statement:} Multi-speaker TTS systems have substantial data requirements, which is challenging for low-resource languages.

  \textbf{Data availability:} \textit{Liepa 2} corpus offers 939 hours of Lithuanian speech, but individual speaker data is sparse (avg: 21 min/speaker).

  \vspace{0.7em}

  \begin{block}{Research aim}
    Measure how varying training dataset \textbf{breadth} (number of speakers) and \textbf{depth} (duration per speaker) affects the synthesis quality of multi-speaker TTS models under a fixed data budget.
  \end{block}

  \vspace{0.7em}

  \textbf{Novelty of the work:}
  \begin{itemize}
    \item First systematic study of ``breadth vs. depth'' trade-offs specifically for the
          Lithuanian language.
    \item Analysis of AR (Tacotron 2) vs. NAR (Glow-TTS) models in low-depth settings
          (7.5 min/speaker).
  \end{itemize}
\end{frame}

% Methods section
\section{Methods}

\begin{frame}{Methodology: Constant ``data budget''}
  Three subsets of the \textit{Liepa 2} corpus are created. To ensure fair comparison, the total training data budget is fixed at \textbf{22.5 hours} for all experiments.

  \vspace{1em}

  \begin{columns}[T]
    \begin{column}{0.33\textwidth}
      \centering
      \textbf{1. Depth}\\
      \vspace{0.5em}
      \fbox{\parbox{0.9\linewidth}{\centering 30 speakers\\ $\times$ \\ 45 min/speaker}}
    \end{column}

    \begin{column}{0.33\textwidth}
      \centering
      \textbf{2. Balance}\\
      \vspace{0.5em}
      \fbox{\parbox{0.9\linewidth}{\centering 60 speakers\\ $\times$ \\ 22.5 min/speaker}}
    \end{column}

    \begin{column}{0.33\textwidth}
      \centering
      \textbf{3. Breadth}\\
      \vspace{0.5em}
      \fbox{\parbox{0.9\linewidth}{\centering 180 speakers\\ $\times$ \\ 7.5 min/speaker}}
    \end{column}
  \end{columns}

  \vspace{1.5em}
  \small{* Speakers are nested (30 $\subset$ 60 $\subset$ 180) and gender-balanced.}
\end{frame}

\begin{frame}{Methodology: Acoustic models}
  Two distinct architectures were trained from scratch on all three data subsets (6 models total).

  \begin{itemize}
    \item \textbf{Tacotron 2:}
          \begin{itemize}
            \item Autoregressive sequence-to-sequence.
            \item High naturalness but slower inference.
          \end{itemize}

    \item \textbf{Glow-TTS:}
          \begin{itemize}
            \item Non-autoregressive, flow-based.
            \item Parallel generation, faster inference.
          \end{itemize}
  \end{itemize}

  \vspace{1em}
  \textbf{Vocoder:} \textit{HiFi-GAN} (pre-trained on VCTK) used for all synthesis to isolate acoustic model performance.
\end{frame}

% Results section
\section{Results}

\begin{frame}{Results: Objective metrics}
  \begin{itemize}
    \item All models converged successfully with clear alignment patterns.
    \item Reducing data per speaker (from 45 min to 7.5 min) had \textbf{minimal impact
            on objective metrics (MCD, $F_0$ RMSE)}.
    \item Tacotron~2 produces more dynamic pitch contours.
  \end{itemize}

  % \vspace{0.5em}

  \begin{figure}[ht]
    \centering
    \subfloat[Tacotron 2 (60 speakers)]{
      \includegraphics[width=0.45\textwidth]{mel_sp_taco2.png}
    }
    \hfill
    \subfloat[Glow-TTS (60 speakers)]{
      \includegraphics[width=0.45\textwidth]{mel_sp_glow.png}
    }
    \caption{Mel-spectrograms generated by Tacotron~2 (left) and Glow-TTS (right) for the same input speaker and text.}
  \end{figure}
\end{frame}

\begin{frame}{Results: Subjective evaluation (MOS)}
  21 native Lithuanian listeners evaluated the naturalness of 6 speakers' synthesized speech.

  \begin{table}[h!]
    \caption{Mean Opinion Score (MOS, 1--5 scale). Higher is better.}
    \centering
    \begin{tabular}{lcc}
      \toprule
      \textbf{Trainset composition} & \textbf{Tacotron 2}      & \textbf{Glow-TTS} \\
      \midrule
      30 spk. $\times$ 45 min       & 3.11 $\pm$ 0.16          & 2.13 $\pm$ 0.12   \\
      60 spk. $\times$ 22.5 min     & \textbf{3.12 $\pm$ 0.17} & 2.17 $\pm$ 0.15   \\
      180 spk. $\times$ 7.5 min     & 3.03 $\pm$ 0.18          & 2.02 $\pm$ 0.14   \\
      \bottomrule
    \end{tabular}
  \end{table}

  \begin{block}{Takeaways:}
    \begin{itemize}
      \item In terms of speech naturalness, \textbf{Tacotron 2} significantly outperforms
            \textbf{Glow-TTS} across all data selection strategies.
      \item There are \textbf{no significant MOS differences} between the three data
            selection strategies for either model.
    \end{itemize}
  \end{block}
\end{frame}

\begin{frame}{Results: MOS by speaker}
  \begin{table}[]
    \centering
    \caption{Average MOS per speaker across all models.}\label{tab:speaker_mos}
    \begin{tabular}{lcc}
      \toprule
      \textbf{Speaker ID} & \textbf{Tacotron~2}      & \textbf{Glow-TTS}        \\
      \midrule
      AS009               & \textbf{4.17 $\pm$ 0.20} & \textbf{2.61 $\pm$ 0.23} \\
      IS031               & 3.26 $\pm$ 0.20          & 2.13 $\pm$ 0.19          \\
      IS038               & 3.48 $\pm$ 0.21          & 2.50 $\pm$ 0.19          \\
      MS052               & 2.26 $\pm$ 0.17          & 1.87 $\pm$ 0.16          \\
      VP131               & 2.43 $\pm$ 0.19          & 1.93 $\pm$ 0.16          \\
      VP427               & 2.92 $\pm$ 0.22          & 1.59 $\pm$ 0.14          \\
      \bottomrule
    \end{tabular}
  \end{table}

  \begin{block}{Takeaways:}
    \begin{itemize}
      \item Certain speakers consistently yield higher/lower MOS across models, indicating
            inherent speaker characteristics may impact synthesis quality.
    \end{itemize}
  \end{block}
\end{frame}

% Remaining Work section
\section{Remaining work}

\begin{frame}{Remaining work}
  \begin{enumerate}
    \item \textbf{Writing final thesis:}
          \begin{itemize}
            \item Corrections based on supervisor feedback.
            \item Post-investigation of speaker-wise differences in MOS\@.
            \item Finalizing the ``Results'', ``Discussion'' chapters.
            \item Plots and tables, formatting, proofreading.
          \end{itemize}

    \item \textbf{Cleaning up the TTS codebase} for publishing in the GitHub repository

    \item Preparing slides, audio samples for the \textbf{final defense presentation}.
  \end{enumerate}
\end{frame}

\begin{frame}{Thank You!}
  \begin{center}
    \vspace{2em}

    \Large Thank you for your attention!
  \end{center}
\end{frame}

\end{document}
